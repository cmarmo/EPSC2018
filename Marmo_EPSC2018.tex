\documentclass[10pt,twocolumn]{article}
\usepackage{times}
\usepackage[utf8x]{inputenc}
\usepackage[final]{pdfpages}
\usepackage{t1enc}
\usepackage{graphicx}
\usepackage{textcomp}
\usepackage[T1]{fontenc}



%%%%%%%%%%%%%%%%%%%%%%%%%%%%%%%%%%%%%%%%%%%%%%%%%%%%%%%%%%%%%%%%%%%%%%%%%%%%%
% Set dimensions of columns, gap between columns, and paragraph indent
%Please do not change/delete any of the settings in the following section as they
%guarantee the correct formatting of your paper.
\setlength{\textheight}{217mm}
\setlength{\textwidth}{160mm}
\setlength{\pdfpagewidth}{180mm}
\setlength{\pdfpageheight}{237mm}

\setlength{\columnsep}{8mm}

\setlength{\topmargin}{0mm}

\setlength{\footskip}{0mm}
\setlength{\headheight}{0mm}
\setlength{\headsep}{0mm}
\setlength{\topskip}{0mm}

\setlength{\voffset}{-15mm}
\setlength{\hoffset}{-15mm}

\newcommand{\Section}[1]{\vspace{-8pt}\section{\hskip -1em.~~#1}\vspace{-3pt}}
\newcommand{\SubSection}[1]{\vspace{-3pt}\subsection{\hskip -1em.~~#1}
      \vspace{-3pt}}

\pagestyle{empty}

%end of dimension settings
%%%%%%%%%%%%%%%%%%%%%%%%%%%%%%%%%%%%%%%%%%%%%%%%%%%%%%%%%%%%%%%%%%%%%%%%%%%%%

\begin{document}


\twocolumn[
\centering{\LARGE \bf A VOEvent model for Meteors and TLEs Optical Detections}
%% Please include your paper title here!

 \vskip10mm

 \begin{flushleft}
{\small C. Marmo (1), M. Garnung (2), P. Le Sidaner (3), B. Cecconi (4), S. Celestin (2) \\
%% Please include author name(s) in this/these line(s)!
%% presenting author (if already known) shall be in bold (e.g. {\bf B. Bman});
(1) GEOPS, Univ. Paris-Sud, CNRS, Univ. Paris-Saclay, Orsay, France, (chiara.marmo@u-psud.fr),
(2) LPC2E, University of Orleans, CNRS, Orleans, France,
(3) DIO, Observatoire de Paris, PSL Research University, CNRS, Paris, France,
(4) LESIA, Observatoire de Paris, Universit\'e PSL, CNRS, Sorbonne Universit\'e, Univ. Paris Diderot, Sorbonne Paris Cit\'e, Meudon, France}
%Please include affiliations here!
\end{flushleft}

%%%%%%%%%%%%%%%%%%%%%
%Please do not change/delete the following vskip command as it
%guarantees the correct distance between abstract information and abstract text.
\vskip10mm %
]
%%%%%%%%%%%%%%%%%%%%%

\thispagestyle{empty}

\section*{Abstract} % Abstract section

Atmospheric transient events are subject of observation and investigation of a growing community.
Professional as well as amateur observation networks are established in Europe and abroad
to collect data important not only for the scientific investigation of some physics processes
but also to assess their possible impacts on Earth’s environment.
We propose here an implementation of the Virtual Observatory Event standard to the domain of
meteors and Transient Luminous Events (TLEs).
A well establish standard for real-time alert on those domains will facilitate coordination between
networks and make easier the extraction of critical information, and will come along with improved
collaboration between the amateur community and agencies (e.g. the ESA Fireball Database, or the CNES
Taranis mission). 

\Section{Introduction} % Introduction
Atmospheric transient events are observed and analyzed for space weather forecasting or for scientific
research in an increasingly systematic way.
In particular meteor and Transient Luminous Events (TLEs) have often been subject of paired 
optical observation campaigns, as both benefit from continuous sky monitoring and similar
hardware installations.

Data provided by professional and amateur observers need to be compared, merged and archived.
A well defined standard describing observation metadata is then necessary in order to efficiently
process those data and enable real-time updates  

\Section{VOEvents for atmospheric surveillance}
VOEvent \cite{voestd} is a standardized protocol developed to report observations of astronomical
events.
It has been officially adopted by the International Virtual Observatory Alliance (IVOA) in 2006.
The VOEvent system is already used by several large-scale projects as the Gamma-Ray Coordinate
System (GCN) \cite{gcn2008}, the Large Synoptic Survey Telescope (LSST),
the European Low Frequency Array (LOFAR), or the Solar Dynamic Observatory (SDO).
VOEvent is used for fast transmission of transient observations.
In the framework of the Planetary Space Weather Service (PSWS) of the Europlanet-H2020 Research
Insfrastructure (EPN2020RI) project \cite{2017EPSC-a}, we propose to
use VOEvent for atmospheric observations like meteors and TLEs.

\SubSection{Meteors}
\cite{efn}
\cite{fripon} \cite{prisma}

\SubSection{TLEs}
Transient Luminous Events (TLEs) are large-scale optical events occurring in the upper-atmosphere 
from the top of thunderclouds up to the  ionosphere.
TLEs may have important effects in local, regional, and global scales of the atmosphere,
but many features of TLEs are not fully understood yet (e.g. \cite{pasko2010}). 
They are sometimes accompanied by terrestrial gamma-ray flashes (TGFs).


The Taranis 1 microsatellite will fly over thousands of TLEs and TGFs for at least two years. Its scientific instruments will be capable of detecting these events and recording their luminous and radiative signatures at high resolution, as well as the electromagnetic perturbations they set off in Earth’s upper atmosphere.

\cite{tleagu}
 the observation of TLEs over unprecedented space and time scales will strongly increase the probability of satellite-ground joint 
detection and hence increase the scientific return of space missions such as ASIM (ESA) and TARANIS (CNES).
\Section{Summary and Perspectives}

\newpage

% Example of an unnumbered section (asterisk at the section command).
\section*{Acknowledgements}
This work benefits from support of VESPA/Europlanet.
The Europlanet 2020 Research Infrastructure is funded by the European Union
under the Horizon 2020 research and innovation program, grant agreement N.654208.
This work is supported by the French Space Agency (CNES) through the satellite mission
TARANIS and a Chair of Excellence to S.C from 10/2012 to 09/2017.

\begin{thebibliography}{}

\small

\bibitem{gcn2008}
Barthelmy, S.: GCN and VOEvent: A status report, Astronomische Nachrichten, Vol. 329, p.340, 2008.
doi:10.1002/asna.200710954

\bibitem{2017EPSC-a}
Cecconi, B., Le Sidaner, P., Andr\'e, N., Marmo, C. and Rasetti, S.: VOEvent for Solar and Planetary Sciences, European Planetary Science Congress, Vol. 11, id. EPSC2017-908, 2017.

\bibitem{fripon}
Colas, F., Zanda, B., Bouley, S., Vaubaillon, J., Marmo, C., Audureau, Y., Kwon, M.~K., Rault, J.~L., Caminade, S., Vernazza, P., Gattacceca, J., Birlan, M., Maquet, L., Egal, A., Rotaru, M., Gruson-Daniel, Y., Birnbaum, C., Cochard, F., Thizy, O.: FRIPON, the French fireball network, European Planetary Science Congress, Vol. 10, id. EPSC2015-800, 2015

%\bibitem{2017EPSC-b}
%Gangloff, M., Andr\'e, N., G\'enot, V., Cecconi, B., Le Sidaner, P., Bouchemit, M., Budnik, E. and Jourdane, N.: Implementation of a Space Weather VOEvent service at IRAP in the frame of Europlanet H2020 PSWS, European Planetary Science Congress, Vol. 11, id. EPSC2017-263, 2017.

\bibitem{prisma}
Gardiol, D., Cellino, A., and Di Martino, M.: PRISMA, Italian network for meteors and atmospheric studies, International Meteor Conference, Egmond, the  Netherlands, pp. 76–79, 2016.

\bibitem{tleagu}
Garnung, M. B., and Celestin, S.: Detecting TLEs using a massive all-sky camera network, AGU Fall Meeting, New Orleans, LA, USA,id. AE23A-2469, 2017.

\bibitem{efn}
Oberst, J., Molau, S., Heinlein, D., Gritzner, C., Schindler, M., Spurny, P., Ceplecha, Z., Rendtel, Betlem, H.: The ``European Fireball Network'': Current status and future prospects, Meteoritics and Planetary Science, Vol. 33, 1998. doi:10.1111/j.1945-5100.1998.tb01606.x

\bibitem{pasko2010}
Pasko, V. P.: Recent advances in theory of transient luminous events, J. Geophys. Res., 115, A00E35, 2010. doi:10.1029/2009JA014860.

\bibitem{voestd}
http://www.ivoa.net/documents/VOEventTransport/20170320/REC-VTP-2.0-20170320.html
\end{thebibliography}
\end{document}
