\documentclass[10pt,twocolumn]{article}
\usepackage{times}
\usepackage[utf8x]{inputenc}
\usepackage[final]{pdfpages}
\usepackage{t1enc}
\usepackage{graphicx}
\usepackage{textcomp}
\usepackage[T1]{fontenc}



%%%%%%%%%%%%%%%%%%%%%%%%%%%%%%%%%%%%%%%%%%%%%%%%%%%%%%%%%%%%%%%%%%%%%%%%%%%%%
% Set dimensions of columns, gap between columns, and paragraph indent
%Please do not change/delete any of the settings in the following section as they
%guarantee the correct formatting of your paper.
\setlength{\textheight}{217mm}
\setlength{\textwidth}{160mm}
\setlength{\pdfpagewidth}{180mm}
\setlength{\pdfpageheight}{237mm}

\setlength{\columnsep}{8mm}

\setlength{\topmargin}{0mm}

\setlength{\footskip}{0mm}
\setlength{\headheight}{0mm}
\setlength{\headsep}{0mm}
\setlength{\topskip}{0mm}

\setlength{\voffset}{-15mm}
\setlength{\hoffset}{-15mm}

\newcommand{\Section}[1]{\vspace{-8pt}\section{\hskip -1em.~~#1}\vspace{-3pt}}
\newcommand{\SubSection}[1]{\vspace{-3pt}\subsection{\hskip -1em.~~#1}
      \vspace{-3pt}}

\pagestyle{empty}

%end of dimension settings
%%%%%%%%%%%%%%%%%%%%%%%%%%%%%%%%%%%%%%%%%%%%%%%%%%%%%%%%%%%%%%%%%%%%%%%%%%%%%

\begin{document}


\twocolumn[
\centering{\LARGE \bf A VOEvent model for Fireball and Sprite Optical Detections}
%% Please include your paper title here!

 \vskip10mm

 \begin{flushleft}
{\small A. Aman (1,2), B. Bman (2) and C. Cman (1) OR A. Aman (1) and/for the
Team-Name\\%% Please include author name(s) in this/these line(s)!
presenting author (if already known) shall be in bold (e.g. {\bf B. Bman});
if there is only one affiliation, do not use any numbering (e.g. A. Aman, B. Bman and C. Cman)\\
(1) Institute of Physics, Alaska, USA, (2) University of Katlenburg-Lindau, Germany (your@address.com / Fax:
+55-555-5555555)} %Please include affiliations here!
\end{flushleft}

%%%%%%%%%%%%%%%%%%%%%
%Please do not change/delete the following vskip command as it
%guarantees the correct distance between abstract information and abstract text.
\vskip10mm %
]
%%%%%%%%%%%%%%%%%%%%%

\thispagestyle{empty}

\section*{Abstract} % Abstract section

This is the abstract section of your paper. Please replace these instructions
with the text of your abstract. The text will appear in two columns. In the
final abstract file (after uploading into Copernicus Office)
each of those two columns are 75 mm wide. If you are including figures, tables and equations,
they MUST be imported into this file. The text will automatically wrap to a second page if necessary.

\Section{Introduction} % Introduction

This is the introduction section of your paper. All section headings are in a large
bold font. All sections and subsections are numbered, respectively. Please
use the Latex command ``$\backslash$Section'' for a numbered section,
``$\backslash$section*'' (with an asterisk) for an unnumbered section and
``$\backslash$SubSection'' for a sub-section. The sections and sub-sections are consecutively numbered.

\SubSection{Sub-Section} % Sub-section

This is the example of a sub-section. As mentioned above, please use the
command ``$\backslash$SubSection\{Your sub-section title\}'' in order to include
your sub-section title in the correct formatting. The sub-sections are also
consecutively numbered.

\Section{An additional section}

You will find an example of how to include your Reference list with the
LaTex bibliography environment at the end of this file. You may cite references with \cite{bib1} or
\cite{bib2}. The reference list should be in an alphabetical order. All references
are put in square brackets and the number in square brackets will appear in
your paper if you use the ``$\backslash$cite\{citation\}'' command. Please note that it is sometimes
necessary to run LaTex twice in order to have the citations be correctly included in the paper.

\Section{Figures} % Figures

You may use any of the common file types, such as .jpg, .tiff, .pdf, etc.
In order to include a figure, please use the LaTex figure environment
as shown in the template.

%\begin{figure}[h]
%\centerline{\includegraphics[width=\columnwidth]{Copernicus_Meetings}}
%\caption{This is the example of an included figure.}
%\label{logo}
%\end{figure}

\Section{Tables} % Tables

You will find a sample of an included table below. Please use the LaTex table
environment in order to include a table.

\begin{table}[h]
\caption{This is the example of an included table.}
\vskip4mm
\centering
\begin{tabular}{lcr}
\hline
Column 1 & Column  2 & Column 3\\
\hline
Line 1 & Line 1 & Line 1\\
Line 2 & Line 2 & Line 2\\
Line 3 & Line 3 & Line 3\\
Line 4 & Line 4 & Line 4\\
Line 5 & Line 5 & Line 5\\
Line 6 & Line 6 & Line 6\\
Line 7 & Line 7 & Line 7\\
Line 8 & Line 8 & Line 8\\
Line 9 & Line 9 & Line 9\\
Line 10 & Line 10 & Line 10\\
Line 11 & Line 11 & Line 11\\
Line 12 & Line 12 & Line 12\\
Line 13 & Line 13 & Line 13\\
Line 14 & Line 14 & Line 14\\
Line 15 & Line 15 & Line 15\\
\hline
\end{tabular}
\end{table}


\Section{Equations} % Equations

Below, you will find examples of two equations. You should use the LaTex
equation environment to include your equation. The equation number is
automatically placed at the right side of the column. The equations are also
consecutively numbered.

\begin{equation}
a^2 + b^2 = c^2
\end{equation}

\begin{equation}
E = m \cdot c^2
\end{equation}

\Section{Summary and Conclusions}

After having finalized your paper in LaTex, please create a respective pdf
file out of the LaTex document. The correct page settings and the formatting
are guaranteed by the preamble which must not be deleted. Please note that you are asked to upload a
pdf file during the abstract submission in Copernicus Office. No other file
type than .pdf is accepted for the file upload. The actual citation header
will be added automatically!

\newpage


% Example of an unnumbered section (asterisk at the section command).
\section*{Acknowledgements}

This is how to do an unnumbered section. In this example, the unnumbered section
is the Acknowledgements section. Here, you may include all persons or
institutions which you would like to thank. We recommend that the abstract is
carefully compiled and thoroughly checked, in particular with regard to the
list of authors, {\bf before} submission.

\begin{thebibliography}{}

\small

\bibitem{bib1}
Author, A., Author, B., and Author, C.: First example of a cited article title, First Example Journal, Vol. 1, pp. 1-100, 1999.

\bibitem{bib2}
Author, D. and Author, E.: Second example of a cited book, Example Publishing House,
2000.

\bibitem{bib3}
Author, F.: Third example of a cited conference paper, The Great
Science Conference, 1--7 February 2001, Sciencetown, Sciencecountry, 2001.

\end{thebibliography}
\end{document}
