\documentclass[10pt,twocolumn]{article}
\usepackage{times}
\usepackage[utf8x]{inputenc}
\usepackage[final]{pdfpages}
\usepackage{t1enc}
\usepackage{graphicx}
\usepackage{textcomp}
\usepackage[T1]{fontenc}



%%%%%%%%%%%%%%%%%%%%%%%%%%%%%%%%%%%%%%%%%%%%%%%%%%%%%%%%%%%%%%%%%%%%%%%%%%%%%
% Set dimensions of columns, gap between columns, and paragraph indent
%Please do not change/delete any of the settings in the following section as they
%guarantee the correct formatting of your paper.
\setlength{\textheight}{217mm}
\setlength{\textwidth}{160mm}
\setlength{\pdfpagewidth}{180mm}
\setlength{\pdfpageheight}{237mm}

\setlength{\columnsep}{8mm}

\setlength{\topmargin}{0mm}

\setlength{\footskip}{0mm}
\setlength{\headheight}{0mm}
\setlength{\headsep}{0mm}
\setlength{\topskip}{0mm}

\setlength{\voffset}{-15mm}
\setlength{\hoffset}{-15mm}

\newcommand{\Section}[1]{\vspace{-8pt}\section{\hskip -1em.~~#1}\vspace{-3pt}}
\newcommand{\SubSection}[1]{\vspace{-3pt}\subsection{\hskip -1em.~~#1}
      \vspace{-3pt}}

\pagestyle{empty}

%end of dimension settings
%%%%%%%%%%%%%%%%%%%%%%%%%%%%%%%%%%%%%%%%%%%%%%%%%%%%%%%%%%%%%%%%%%%%%%%%%%%%%

\begin{document}


\twocolumn[
\centering{\LARGE \bf A VOEvent model for Meteors and TLEs Optical Detections}
%% Please include your paper title here!

 \vskip10mm

 \begin{flushleft}
{\small C. Marmo (1), M. Garnung (2), P. Le Sidaner (3), B. Cecconi (4), S. Celestin (2) \\
%% Please include author name(s) in this/these line(s)!
%% presenting author (if already known) shall be in bold (e.g. {\bf B. Bman});
(1)  (2) (3) (4)}
%Please include affiliations here!
\end{flushleft}

%%%%%%%%%%%%%%%%%%%%%
%Please do not change/delete the following vskip command as it
%guarantees the correct distance between abstract information and abstract text.
\vskip10mm %
]
%%%%%%%%%%%%%%%%%%%%%

\thispagestyle{empty}

\section*{Abstract} % Abstract section

Atmospheric transient events are subject of observation and investigation of a growing community.
Professional as well as amateur observation networks are established in Europe and abroad
to collect data important not only for the scientific investigation of some physics processes
but also to assess their possible impacts on Earth’s environment.
We propose here an implementation of the Virtual Observatory Event standard to the domain of
meteors and Transient Luminous Events (TLEs).
A well establish standard for real-time alert on those domains will facilitate coordination between
networks and make easier the extraction of critical information.
Improving collaboration between the amateur community and agencies (eg the ESA Fireball Database,
or the CNES Taranis mission). 

\Section{Introduction} % Introduction

\cite{2017EPSC-a}

\SubSection{Sub-Section} % Sub-section

\Section{An additional section}

\Section{Figures} % Figures

You may use any of the common file types, such as .jpg, .tiff, .pdf, etc.
In order to include a figure, please use the LaTex figure environment
as shown in the template.

%\begin{figure}[h]
%\centerline{\includegraphics[width=\columnwidth]{Copernicus_Meetings}}
%\caption{This is the example of an included figure.}
%\label{logo}
%\end{figure}

\Section{Tables} % Tables

You will find a sample of an included table below. Please use the LaTex table
environment in order to include a table.

\begin{table}[h]
\caption{This is the example of an included table.}
\vskip4mm
\centering
\begin{tabular}{lcr}
\hline
Column 1 & Column  2 & Column 3\\
\hline
Line 1 & Line 1 & Line 1\\
Line 2 & Line 2 & Line 2\\
Line 3 & Line 3 & Line 3\\
Line 4 & Line 4 & Line 4\\
Line 5 & Line 5 & Line 5\\
Line 6 & Line 6 & Line 6\\
Line 7 & Line 7 & Line 7\\
Line 8 & Line 8 & Line 8\\
Line 9 & Line 9 & Line 9\\
Line 10 & Line 10 & Line 10\\
Line 11 & Line 11 & Line 11\\
Line 12 & Line 12 & Line 12\\
Line 13 & Line 13 & Line 13\\
Line 14 & Line 14 & Line 14\\
Line 15 & Line 15 & Line 15\\
\hline
\end{tabular}
\end{table}


\Section{Equations} % Equations

Below, you will find examples of two equations. You should use the LaTex
equation environment to include your equation. The equation number is
automatically placed at the right side of the column. The equations are also
consecutively numbered.

\begin{equation}
a^2 + b^2 = c^2
\end{equation}

\begin{equation}
E = m \cdot c^2
\end{equation}

\Section{Summary and Conclusions}

\newpage

% Example of an unnumbered section (asterisk at the section command).
\section*{Acknowledgements}

This work benefits from support of VESPA/Europlanet.
The Europlanet 2020 Research Infrastructure is funded by the European Union
under the Horizon 2020 research and innovation program, grant agreement N.654208.
This work is supported by the French Space Agency (CNES) through the satellite mission
TARANIS and a Chair of Excellence to S.C from 10/2012 to 09/2017.

\begin{thebibliography}{}

\small

\bibitem{2017EPSC-a}
Cecconi, B., Le Sidaner, P., Andr\'e, N., Marmo, C. and Rasetti, S.: VOEvent for Solar and Planetary Sciences, European Planetary Science Congress, Vol. 11, id. EPSC2017-908, 2017.

\bibitem{2017EPSC-b}
Gangloff, M., Andr\'e, N., G\'enot, V., Cecconi, B., Le Sidaner, P., Bouchemit, M., Budnik, E. and Jourdane, N.: Implementation of a Space Weather VOEvent service at IRAP in the frame of Europlanet H2020 PSWS, European Planetary Science Congress, Vol. 11, id. EPSC2017-263, 2017.

\end{thebibliography}
\end{document}
